\Conclusion % заключение к отчёту

В процессе разработки системы обнаружения целей, важным этапом является выбор оптимальной схемы объединения данных от нескольких РЛС. Она должна обеспечивать максимальную вероятность истинного обнаружения цели при заданной вероятности ложного обнаружения. 

Увеличение числа обнаружений не менее M из N приводит к уменьшению вероятности ложной тревоги и увеличению вероятности пропуска цели. Однако, выбор конкретной схемы объединения может повлиять на эти вероятности. Например, схема «ИЛИ» обеспечивает большую зону видимости, но имеет проблемы с вероятностью ложной тревоги. С другой стороны, схема «И» имеет низкую вероятность правильного обнаружения, что приводит к уменьшению зоны видимости.

В случае использования схемы «ИЛИ», увеличение числа РЛС может привести к увеличению зоны видимости системы. Однако, при этом может возникнуть проблема с вероятностью ложной тревоги. В то же время, использование схемы «И» может привести к катастрофически низкой вероятности правильного обнаружения цели, что снижает зону видимости. 

Таким образом, выбор оптимальной схемы объединения данных является важным шагом в разработке системы обнаружения целей. При этом необходимо учитывать как требования по вероятности истинного обнаружения и ложного обнаружения, так и возможные проблемы с зоной видимости при использовании конкретной схемы.