\chapter{Теоретические сведения}
\label{cha:impl}
Для сигнала со случайной амплитудой, распределенной по закону Релея, и 
равномерно распределенной начальной фазой вероятность правильного обнаружения: 
$$ D = F^{\frac{1}{1 + 0.5 Q^2}} $$
где D – вероятность правильного обнаружения, а F – вероятность ложной тревоги. 
Величина 
$$ Q = \sqrt{\frac{2E_{res}}{N_0}} $$
численно равная отношению сигнал-помеха по напряжению на выходе согласованного, 
называется параметром обнаружения или отношением сигнал/шум. 
Найдем сначала найдем энергию на входе приемника $E_{res}$. Для этого рассчитаем 
плотности потока энергии у цели, удаленной на расстояние $r_1$ от передающей антенны: 
$$ I_t = \frac{P_p G_{tr}(\alpha, \beta) }{4 \pi r_{1}^2} $$
где $P_p$ – импульсная мощность. $G_{tr}$ – коэффициент усиления передающей антенны. 
Реальные передающие антенны принято сравнивать с идеальной ненаправленной 
антенной. Плотность потока мощности у таких идеальных антенн на расстоянии r от центра 
антенны составит:
$$I_0 = \frac{P}{4 \pi r^2}$$
где $P$ – подводимая к антенне мощность.
Эффективная площадь приема 
и коэффициент усиления антенны взаимосвязаны: 
$$G = \frac{4 \pi A}{\lambda^2}$$

Формулу мощности принимаемого сигнала 
можно записать так: 
$$ P_{res} = \frac{P_p G_{tr} \sigma A_{res} }{(4 \pi)^2 r^4} $$

$$ E_{res} = \frac{P_p \tau_p G^2 \sigma \lambda^2 }{(4 \pi)^3 (r^2 + h^2)^2} $$