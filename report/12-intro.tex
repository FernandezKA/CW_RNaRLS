\Introduction

Данный документ представляет собой отчет к курсовой расчетной работе по дисциплине "Радиолокационные и радионавигационные системы". 
В ходе данной работы было выполнено следующее: 
\begin{itemize}
    \item проведено ознакомление с методами объединения информации, 
    \item построена зависимость вероятности правильного обнаружения в зависимости от дальности для одной РЛС, 
    \item оценен процент площади, на которой происходит обнаружение с вероятностью 95 \% для одиночной РЛС, расположенной в центре области, 
    \item обеспечено обнаружение с вероятностями правильного обнаружения $D_0 = 95 \%$, вероятностью ложной тревоги $F_0 = 10^{-8}$ в заданной области, 
    \item для обеспечения данной конфигурации были выбраны количество и конфигурация РЛС, 
    \item  выбран критерий обнаружения в комплексе, 
    \item  построены зоны обнаружения для выбранной конфигурации РЛС без объединения, при объединении по схемам «И», «ИЛИ» и выбранному критерию обнаружения. 
\end{itemize} 

В ходе данной работы был использован вариант №24. 